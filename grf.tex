\documentclass{beamer}
\usepackage{multicol}
\usepackage{graphicx}
\usepackage[english]{babel}
\usepackage{amsmath}
\usepackage{amssymb}
\usepackage{mathtools}
\usepackage{mathrsfs}
\usepackage{physics}
\usepackage{bm}
\usepackage{bbm}
\usetheme[alternativetitlepage=bild]{Rub}
\renewcommand{\vec}[1]{\bm{#1}}
\begin{document}
\title{Gaussian Random Field Generation \\for stochastic PDEs}   
\author{Timo Schorlepp} 
\date{\today} 
\titlegraphic{titel.png}
\frame{\titlepage} 

\frame{\frametitle{Table of contents}\tableofcontents} 

%%%%%%%%%%%%%%%%%%%%%%%%%%%%%%%%%%%%%%%%%%%%%%%%%
\section{Motivation} 
%%%%%%%%%%%%%%%%%%%%%%%%%%%%%%%%%%%%%%%%%%%%%%%%%
\frame{\frametitle{Motivation:\\Exemplary applications} 
\begin{columns}[t]
\column{.5\textwidth}
\centering
\includegraphics[width=.65\textwidth]{figures/cosmogrf}\\
\tiny{
Vogelsberger et al. 2014, Illustris simulation}\\
\includegraphics[width=.65\textwidth]{figures/2dns}\\
\tiny{Murray 2017, 2$D$ stochastic NSE vorticity}
\column{.5\textwidth}
\centering
\includegraphics[width=.65\textwidth]{figures/kpz}\\
\tiny{Kuennen, Wang 2008, KPZ surface growth}\\[.2cm]
\includegraphics[width=.65\textwidth]{figures/aquifer}\\
\tiny{Wikimedia, aquifer sketch}
\end{columns}
}
%%%%%%%%%%%%%%%%%%%%%%%%%%%%%%%%%%%%%%%%%%%%%%%%%
\frame{\frametitle{Motivation:\\Mathematical examples} 
Deterministic incompressible NSE for $\vec{u}:\mathbb{T}^n \times \mathbb{R}_+ \to \mathbb{R}^n$:
\begin{align*}
\partial_t \vec{u} + (\vec{u} \cdot \nabla) \vec{u} + \nabla p = \nu \Delta \vec{u}\; , \; \nabla \cdot \vec{u} = 0 
\end{align*}
Without forcing: 
\begin{align*}
\dv{}{t} \int \mathrm{d}V \; \frac{1}{2} \abs{\vec{u}}^2 = - 2 \nu \int \mathrm{d}V \; \trace((\nabla \otimes \vec{u})^T (\nabla \otimes \vec{u})) \leq -2 C \nu \int \mathrm{d}V \; \abs{\vec{u}}^2
\end{align*}
Gronwall: Energy decays exponentially, need forcing for interesting long-term behavior, e.g. Gaussian forcing with homogeneous, isotropic correlation matrix!\\
\medskip
Similarly: Stochastic heat equation
}
%%%%%%%%%%%%%%%%%%%%%%%%%%%%%%%%%%%%%%%%%%%%%%%%%
\section{Theory} 
%%%%%%%%%%%%%%%%%%%%%%%%%%%%%%%%%%%%%%%%%%%%%%%%%
\frame{\frametitle{Theory:\\Basic Definitions}
\begin{block}{Definition 1}
A random field $\vec{\xi}$ is an indexed family of random variables
\begin{align*}
\vec{\xi} = \left\{ \vec{\xi}(\vec{x}) : \Omega \to \mathbb{R}^m \; ; \; \vec{x} \in T \subseteq \mathbb{R}^d \right\}.
\end{align*}
\end{block}
\begin{alertblock}{Remark 2}
\begin{itemize}
\item Generalization of stochastic processes
\item No details on Kolmogorov Existence Theorem etc. here
\end{itemize}
\end{alertblock}
}

\frame{\frametitle{Theory:\\Basic Definitions}
\begin{block}{Definition 3}
A random field $\vec{\xi}$ is called Gaussian iff $\forall k \in \mathbb{N}: \forall \{\vec{x}^{(0)}, \cdots , \vec{x}^{(k-1)} \} \subseteq T: \vec{\xi}(\vec{x}^{(0)}) =: \vec{\xi}^{(0)}, \cdots \vec{\xi}(\vec{x}^{(k-1)})=: \vec{\xi}^{(k-1)}$ are jointly normally distributed, i.e.
\begin{align*}
p_{\vec{\xi}^{(0)}, \cdots ,\vec{\xi}^{(k-1)}}(\vec{y}^{(0)}, \cdots \vec{y}^{(k-1)}) = &\det(2 \pi \Sigma)^{- \frac{1}{2}} \cdot \\ & \cdot \exp(-\frac{1}{2} (\vec{y}-\vec{\mu})^T \Sigma^{-1} (\vec{y}-\vec{\mu}))
\end{align*}
where $\Sigma$ is the $km \times km$ covariance matrix 
\begin{align*}
\Sigma_{m r + i,m s + j} = \text{Cov}\left(\xi^{(r)}_i,\xi^{(s)}_j \right) =: \chi_{ij}\left(\vec{x}^{(r)}, \vec{x}^{(s)} \right)
\end{align*}
and $\vec{\mu}$ is the $km$-dim.\ mean vector $\mu_{mr + i} = \left< \xi^{(r)}_i \right>$.
\end{block}
}

\frame{\frametitle{Theory:\\Basic Definitions}
\begin{alertblock}{Remark 4}
\begin{itemize}
\item Gaussian random fields (GRF) are completely specified by $\vec{\mu}(\vec{x})$ and $\chi \left(\vec{x}',\vec{x}\right)$ $\Longrightarrow$ easy!
\item We will assume $\vec{\mu}(\vec{x}) \equiv 0$ wlog in the following (also necessary for isotropy) as well as $d=m$
\item $\chi$ needs to be positive semidefinite for all $\vec{x}, \vec{x}'$ since $\vec{y}^T \chi \left(\vec{x}',\vec{x}\right) \vec{y} = \text{Var}\left(\vec{y}^T \left(\vec{\xi}' + \vec{\xi} \right) \right) \geq 0$
\end{itemize}
\end{alertblock}
}

\frame{\frametitle{Theory:\\Basic Definitions}
\begin{exampleblock}{Example/Definition 5}
\begin{itemize}
\item White noise $\chi_{ij}\left(\vec{x}',\vec{x}\right) = \delta(x'_i-x_j)$
\item Homogeneous (translation-invariant, stationary), isotropic ($O(n)$-invariant) and solenoidal correlation tensor:
\begin{align*}
\chi_{ij}(\vec{x}', \vec{x}) = \chi_{ij}(\vec{x}'-\vec{x}) = \chi_{ij}(\vec{r}) = f(r) \delta_{ij} + \frac{r f'(r)}{d-1} \left( \delta_{ij}-\frac{r_i r_j}{r^2}\right)
\end{align*}
\end{itemize}
\end{exampleblock}


}

\section{Numerical implementation}
\frame{\frametitle{Title}
.
}


\section{Results} 
\frame{\frametitle{Title}
.
}


\section{Conclusion}
\frame{\frametitle{Title}

\begin{block}{A}
bla
\end{block}

\begin{exampleblock}{B}
bla
\end{exampleblock}


\begin{alertblock}{C}
bla
\end{alertblock}
}

\end{document}

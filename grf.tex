\documentclass{beamer}
\usepackage{multicol}
\usepackage{graphicx}
\usepackage[english]{babel}
\usepackage{amsmath}
\usepackage{amssymb}
\usepackage{mathtools}
\usepackage{mathrsfs}
\usepackage{physics}
\usepackage{bm}
\usepackage{bbm}
\usetheme[alternativetitlepage=bild]{Rub}
\renewcommand{\vec}[1]{\bm{#1}}
\begin{document}
\title{Gaussian Random Field Generation \\for stochastic PDEs}   
\author{Timo Schorlepp} 
\date{\today} 
\titlegraphic{titel.png}
\frame{\titlepage} 

\frame{\frametitle{Table of contents}\tableofcontents} 


\section{Motivation} 
\frame{\frametitle{Motivation I} 
\begin{columns}[t]
\column{.5\textwidth}
\centering
\includegraphics[width=.65\textwidth]{figures/cosmogrf}\\
\tiny{
Vogelsberger et al. 2014, Illustris simulation}\\
\includegraphics[width=.65\textwidth]{figures/2dns}\\
\tiny{Murray 2017, 2$D$ stochastic NSE vorticity}
\column{.5\textwidth}
\centering
\includegraphics[width=.65\textwidth]{figures/kpz}\\
\tiny{Kuennen, Wang 2008, KPZ surface growth}\\[.2cm]
\includegraphics[width=.65\textwidth]{figures/aquifer}\\
\tiny{Wikimedia, aquifer sketch}
\end{columns}
}

\frame{\frametitle{Motivation II} 
Deterministic incompressible NSE for $\vec{u}:\mathbb{T}^n \times \mathbb{R}_+ \to \mathbb{R}^n$:
\begin{align*}
\partial_t \vec{u} + (\vec{u} \cdot \nabla) \vec{u} + \nabla p = \nu \Delta \vec{u}\; , \; \nabla \cdot \vec{u} = 0 
\end{align*}
Without forcing: 
\begin{align*}
\dv{}{t} \int \mathrm{d}V \; \frac{1}{2} \abs{\vec{u}}^2 = - 2 \nu \int \mathrm{d}V \; \trace((\nabla \otimes \vec{u})^T (\nabla \otimes \vec{u})) \leq -2 C \nu \int \mathrm{d}V \; \abs{\vec{u}}^2
\end{align*}
Gronwall: Energy decays exponentially, need forcing for interesting long-term behavior, e.g. Gaussian forcing with homogeneous, isotropic correlation matrix!\\
\medskip
Similarly: Stochastic heat equation
}
\section{Theory} 
\frame{\frametitle{Basic Definitions}
. \\
\pause
.
}

\section{Numerical implementation}
\frame{\frametitle{Title}
.
}


\section{Results} 
\frame{\frametitle{Title}
.
}


\section{Conclusion}
\frame{\frametitle{Title}

\begin{block}{A}
bla
\end{block}

\begin{exampleblock}{B}
bla
\end{exampleblock}


\begin{alertblock}{C}
bla
\end{alertblock}
}

\end{document}
